\documentclass[a4paper,12pt]{article}

\usepackage{graphicx}
\usepackage{caption}
\usepackage{subcaption}
\usepackage{tikz}
\usepackage{pgf}
\usepackage{amsmath}
\usepackage{amssymb}
\usetikzlibrary{arrows.meta}
\usepackage[utf8]{inputenc}
\usepackage[english,greek]{babel}
\usepackage{hyperref}

\title{Εργασία Μέρος Β - Θεωρία Εκτίμησης και Ανίχνευσης \\ Ομάδα 29}
\author{Ρουσομάνης Γεώργιος (ΑΕΜ: 10703) \\ Αριστείδης Δασκαλόπουλος (10640)}
\date{Μάιος 2025}

\begin{document}

\maketitle

\section*{Εύρεση φίλτρου \selectlanguage{english}Wiener\selectlanguage{greek}}

Για την αποθορυβοποίηση των δεδομένων θα εξετάσουμε δύο μεθόδους. Στην πρώτη 
(\selectlanguage{english}offline\selectlanguage{greek}) μέθοδο, θα προσπαθήσουμε να 
εκτιμήσουμε τα καθαρά δυναμικά 
$\theta = [\mathbf{v}[0], \mathbf{v}[1], \ldots, \mathbf{v}[n-1]]$
με βάση τις τρέχουσες, παρελθοντικές και μελλοντικές τιμές των μετρήσεων 
$[\mathbf{y}[0], \mathbf{y}[1], \ldots, \mathbf{y}[n-1]]$ 
-- \selectlanguage{english}smoothing\selectlanguage{greek}. Στη δεύτερη 
(\selectlanguage{english}offline\selectlanguage{greek}) μέθοδο, θα εκτιμήσουμε το 
$\theta = \mathbf{v}[m]$ με βάση μόνο τις τρέχουσες και παρελθοντικές τιμές των μετρήσεων 
$[\mathbf{y}[0], \mathbf{y}[1], \ldots, \mathbf{y}[m-1]]$
-- \selectlanguage{english}filtering\selectlanguage{greek}.

Για κάθε μία από τις παραπάνω μεθόδους, ακολουθούμε δύο προσεγγίσεις. Στην πρώτη (μονοκαναλική) προσέγγιση, 
κάθε κανάλι αντιμετωπίζεται ανεξάρτητα από τα υπόλοιπα. Στη δεύτερη (πολυκαναλική) προσέγγιση, λαμβάνεται 
υπόψη η συσχέτιση μεταξύ των καναλιών – υπενθυμίζουμε ότι ο θόρυβος που οφείλεται στο ανοιγοκλείσιμο των 
ματιών είναι εντονότερος στα μετωπιαία ηλεκτρόδια.

\subsection*{Μονοκαναλικό \selectlanguage{english}Smoothing\selectlanguage{greek}}
Έστω, $\mathbf{y}_i, \, \mathbf{d}_i, \, \mathbf{v}_i$
οι μετρήσεις, ο θόρυβος και το καθαρό δυναμικό στο $i$-οστό κανάλι τις χρονικές στιγμές $i=0,...,n-1$. 
Η εκτίμηση του φίλτρου \selectlanguage{english}Wiener\selectlanguage{greek} είναι:
\[
\hat{\mathbf{v}}_i = \mathbf{C}_{vy}^{(i)}\left(C_{yy}^{(i)}\right)^{-1}\mathbf{y}_i, \quad i = 1,...,N
\]
όπου $\mathbf{y}_i = [y_i[0], y_i[1], \ldots, y_i[n-1]]$ οι μετρήσεις του $i$-οστού καναλιού,
$\mathbf{C}_{vy}^{(i)}$ ο πίνακας συνδιασποράς των $\mathbf{v}_i, \, \mathbf{y}_i$ και $C_{yy}^{(i)}$
ο πίνακας αυτοδιασποράς του $\mathbf{y}_i$.

Αν θεωρήσουμε ότι τα $\mathbf{y}_i, \, \mathbf{v}_i, \, \mathbf{d}_i$ είναι σήματα μηδενικής μέσης τιμής
και τα $\mathbf{v}_i, \, \mathbf{d}_i$ είναι ανεξάρτητα τότε έχουμε:
\[
\begin{aligned}
    \mathbf{C}_{yy}^{(i)} &= \mathbb{E}[\mathbf{y}_i\mathbf{y}_i^T] = \mathbf{R}_{yy}^{(i)} \\
    \mathbf{C}_{vy}^{(i)} &= \mathbb{E}[\mathbf{v}_i\mathbf{y}_i^T] = 
    \mathbb{E}[\mathbf{v}_i(\mathbf{d}_i + \mathbf{v}_i)^T] = \mathbf{R}_{vv}^{(i)}
\end{aligned}
\]
με $\mathbf{R}_{yy}^{(i)}, \, \mathbf{R}_{vv}^{(i)}$ τους πίνακες αυτοδιακύμανσης των χρονοσειρών
$\mathbf{y}_i, \, \mathbf{v}_i$ αντίστοιχα.

Στα δεδομένα εκπαίδευσης, γνωρίζουμε εκ των προτέρων τις χρονικές στιγμές όπου στις μετρήσεις μας υπάρχει
θόρυβος οι οποίες μάλιστα είναι κοινές για όλα τα κανάλια. Από το Σχήμα~\ref{}

\end{document}
