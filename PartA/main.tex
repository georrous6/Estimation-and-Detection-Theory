\documentclass{beamer}
\usetheme{Madrid}
\usecolortheme{seahorse}
\usepackage{amsmath, amssymb}

\title[Minimum Variance in Biased Estimation]{Minimum Variance in Biased Estimation: Bounds and Asymptotically Optimal Estimators}
\author{Yonina C. Eldar \\ Presented by: Aristeidis Daskalopoulos (10640) \\ Rousomanis Georgios (10703)}
\institute{IEEE Transactions on Signal Processing, July 2004}
\date{}

\begin{document}

\begin{frame}
  \titlepage
\end{frame}

\begin{frame}{Motivation and Background}
  \begin{itemize}
    \item A common approach to developing well-behaved estimators
    in overparameterized estimation problems is to use \textit{regularization} techniques.
    \item Regularization reduces variance but introduces bias.
    \item Cramér–Rao Lower Bound (CRLB) assumes unbiased estimators.
    \item Need for bounds applicable to biased estimators.
  \end{itemize}
\end{frame}

\begin{frame}{Key Goals of the Paper}
  \begin{itemize}
    \item Develop Uniform Cramér–Rao Lower Bound (UCRLB) for biased estimators.
    \item Use Frobenius and spectral norms of the bias gradient matrix.
    \item Construct estimators (Tikhonov, shrunken, PML) that achieve the bounds.
  \end{itemize}
\end{frame}

\begin{frame}{Classical and Biased CRLB}
  \begin{itemize}
    \item Unbiased CRLB: $\operatorname{Var}(\hat{\mathbf{x}}) \geq \mathbf{J}^{-1}$
    \item Biased CRLB:
    \begin{equation*}
        \begin{aligned}
            \mathbf{b}(\mathbf{x}_0) &= \mathbb{E}[\hat{\mathbf{x}}] - \mathbf{x}_0 \\
            \operatorname{Cov}(\hat{\mathbf{x}}) &\geq (\mathbf{I} + \mathbf{D}) \mathbf{J}^{-1} 
            (\mathbf{I} + \mathbf{D})^* \triangleq \mathbf{C}(\mathbf{D}) \\
            \mathbf{D} &= \frac{\partial \mathbf{b}(\mathbf{x}_0)}{\partial \mathbf{x}}
        \end{aligned}
    \end{equation*}
    \item Biased CRLB does not depend directly on the bias but only on the bias gradient matrix!
    \item $D$ is invariant to a constant bias term so that in effect, it characterizes the part of the bias 
    that cannot be removed
  \end{itemize}
\end{frame}

\begin{frame}{Selecting Bias Gradient Matrix}
  \begin{itemize}
  \item Given a desired bias gradient, the biased CRLB serves as a bound on the smallest attainable variance.
  \item How to choose $\mathbf{D}$?
  \item Instead, use norms of $\mathbf{D}$ to constrain variance
  \begin{itemize}
    \item Frobenius norm (average bias)
    \item Spectral norm (worst-case bias)
  \end{itemize}
  \item Uniform CRLB (UCRLB): is a bound on the smallest attainable
  variance that can be achieved using any estimator with bias gradient whose norm is bounded by a constant.
  \end{itemize}
\end{frame}

\begin{frame}{Worst Case Bias Constraint}
  \begin{itemize}
    \item Generally, minimizing the bias results in an increase in variance and vice versa.
    \item Tradeoff between bias and variance
    \item Minimize $\operatorname{Tr}[\mathbf{C}(\mathbf{D})]$ subject to some constraint on $\mathbf{D}$.
    \item How to develop a meaningful constraint on $\mathbf{D}$?
    
    First-order Taylor expansion at the neighbor of $\mathbf{x}_0$:
    \[
        \mathbf{b}(\mathbf{x}) - \mathbf{b}(\mathbf{x}_0) \approx \mathbf{D}(\mathbf{x} - \mathbf{x}_0) 
        \triangleq \mathbf{D} \mathbf{u}
    \]
    thus,
    \[
        ||\mathbf{b}(\mathbf{x}) - \mathbf{b}(\mathbf{x}_0)||^2 \approx 
        \mathbf{u}^* \mathbf{D}^* \mathbf{D u} 
        \triangleq \mathcal{V}
    \]
    Let
    \[
    \mathcal{S} = 
    \{\mathbf{x}|(\mathbf{x} - \mathbf{x}_0)^*\mathbf{M}^{-1}(\mathbf{x} - \mathbf{x}_0) \leq 1\}, 
    \quad \mathbf{M} > 0
    \]
    
  \end{itemize}
\end{frame}

\begin{frame}{Worst Case Bias Constraint (cont.)}
The maximal variation of the bias norm over the region $\mathcal{S}$ is
\[
\max_{\mathbf{u} \in \mathcal{S}} \mathcal{V} = 
\max_{\mathbf{u}^*\mathbf{M}^{-1}\mathbf{u} \leq 1} ||\mathbf{D u}||^2
\]
Let $\mathbf{z} = \mathbf{M}^{-1/2}\mathbf{u}$, then
\[
\max_{\mathbf{u}^*\mathbf{M}^{-1}\mathbf{u} \leq 1} ||\mathbf{D u}||^2 = 
\max_{\mathbf{z}^*\mathbf{z} \leq 1} ||\mathbf{D} \mathbf{M}^{1/2}\mathbf{z}||^2 = 
\max_{\mathbf{z}^*\mathbf{z} \leq 1} \mathbf{z}^* \mathbf{M}^{1/2} \mathbf{D}^* \mathbf{D} 
\mathbf{M}^{1/2} \mathbf{z}
\]
and finally
\[
\max_{\mathbf{u}^*\mathbf{M}^{-1}\mathbf{u} \leq 1} ||\mathbf{D u}||^2 = 
||\mathbf{D} \mathbf{M}^{1/2}||^2
\]
where $|| \cdot ||$ denote the \textit{spectral norm} of a matrix. Thus,
\[
D_{WC} = 
\max_{\mathbf{z} \in \mathbb{C}^m, \, ||\mathbf{z}|| = 1} \mathbf{z}^* \mathbf{S} \mathbf{D}^* \mathbf{D} 
\mathbf{S} \mathbf{z}, \quad S \geq 0
\]
\end{frame}

\begin{frame}{Average Bias Constraint}
\begin{itemize}
    \item The worst-case variation occurs when $\mathbf{z}$ is chosen to be a unit-norm vector in the direction
    of the eigenvector corresponding to the largest eigenvalue of 
    $\mathbf{M}^{1/2} \mathbf{D}^* \mathbf{D} \mathbf{M}^{1/2}$.
    \item How to develop an average bias measure?
    \item Choose $\mathbf{z}$ as a linear combination of the eigenvectors of 
    $\mathbf{M}^{1/2} \mathbf{D}^* \mathbf{D} \mathbf{M}^{1/2}$ such that 
    $||\mathbf{z}||=1$:
    \[
        \mathbf{z} = \sum_{i=1}^m \alpha_i \mathbf{v}_i, \quad \sum_{i=1}^m \alpha_i^2 = 1
    \]
    Then
    \[
        \mathcal{V} = \mathbf{z}^* \mathbf{M}^{1/2} \mathbf{D}^* \mathbf{D} \mathbf{M}^{1/2} \mathbf{z} =
        \sum_{i=1}^m \alpha_i^2 \lambda_i
    \]
    where $\lambda_i$ the eigenvalues of $\mathbf{M}^{1/2} \mathbf{D}^* \mathbf{D} \mathbf{M}^{1/2}$.
\end{itemize}
\end{frame}

\begin{frame}{Average Bias Constraint (cont.)}
Let
\begin{itemize}
    \item $A$ the diagonal matrix with diagonal elements $\alpha_{ii}$
    \item $\mathbf{V}$ the matrix of eigenvectors $\mathbf{v}_i$
    \item $\mathbf{Q} = \mathbf{M}^{1/2} \mathbf{V A V}^* \mathbf{M}^{1/2}$
\end{itemize}
Then 
\[
 \mathcal{V} = \operatorname{Tr}\left(\mathbf{V A V}^*\mathbf{M}^{1/2}\mathbf{D}^*\mathbf{DM}^{1/2}\right)
  = \operatorname{Tr}(\mathbf{D}^*\mathbf{D Q})
\]
It follows that the weighted Frobenius norm $\operatorname{Tr}(\mathbf{D}^*\mathbf{D Q})$ of $\mathbf{D}$ 
is a measure of the average variation in the norm of the bias over the ellipsoid and is therefore a reasonable
average bias measure. Thus:
\[
D_{AVG} = \operatorname{Tr}(\mathbf{D}^*\mathbf{D W}), \quad \mathbf{W} \geq 0
\]
\end{frame}

\begin{frame}{UCRLB with Average Bias Constraint}
    Consider the problem of minimizing $C(\mathbf{D})$ subject to
    \[
    D_{AVG} = \operatorname{Tr}(\mathbf{D}^*\mathbf{D W}) \leq \gamma
    \]
    \begin{itemize}
        \item If $\gamma \geq \operatorname{Tr}(\mathbf{W})$, then we choose 
        $\mathbf{D} = -\mathbf{I} \Rightarrow C(\mathbf{D}) = 0$.
        \item If $\gamma < \operatorname{Tr}(\mathbf{W})$, consider the Lagrangian:
        \[
        L = (\mathbf{I} + \mathbf{D}) \mathbf{J}^{-1} (\mathbf{I} + \mathbf{D})^* +
        \alpha (\operatorname{Tr}(\mathbf{D}^*\mathbf{D W}) - \gamma), \quad \alpha \leq 0
        \]
        Take derivative w.r.t. $\mathbf{D}$ and set to zero:
        \[
        (\mathbf{I} + \mathbf{D})\mathbf{J}^{-1} + \alpha\mathbf{D}\mathbf{W} = 0
        \]
        Solution:
        \[
        \hat{\mathbf{D}}_{AVG} = (\mathbf{I} + \alpha \mathbf{W J})^{-1}
        \]
    \end{itemize}
\end{frame}

%%%%%%%%%%%%%%%%%%%%%%%%%%%%%%%%%%%%%%%%%%%%%%%%%%%%%%%%%%%%%%%%%%%%%%%%%%%%%%%%%%%%%%%

\begin{frame}{UCRLB with Average Bias Constraint (cont.)}
    \[
    \hat{\mathbf{D}}_{AVG} = (\mathbf{I} + \alpha \mathbf{W J})^{-1}
    \]
    \begin{itemize}
        \item $\alpha = 0 \Rightarrow \hat{\mathbf{D}}_{AVG} = -\mathbf{I}$ which violates the constraints.
        \item If $\alpha > 0$, then from KKT: 
        $\operatorname{Tr}(\mathbf{D}_{AVG}^*\mathbf{D}_{AVG} \mathbf{W}) = \gamma$.
        
        When $\mathbf{W} > 0$, we can write:
        \[
        \text{Tr}\left((\mathbf{W}^{-1} + \alpha\mathbf{J})^{-2}\mathbf{W}^{-1}\right) = \gamma
        \]
        and variance bound becomes:
        \[
        C \geq \alpha^2\text{Tr}\left((\mathbf{W}^{-1} + \alpha\mathbf{J})^{-2}\mathbf{J}\right)
        \]
    \end{itemize}
\end{frame}

%%%%%%%%%%%%%%%%%%%%%%%%%%%%%%%%%%%%%%%%%%%%%%%%%%%%%%%%%%%%%%%%%%%%%%%%%%%%%%%%%%%%%

\begin{frame}{Comparison with UCRLB}
\begin{itemize}
    \item Until know we minimized the joint variance
    \item Scalar UCRLB minimizes variance for each component separately:
    \[
    \left[\mathbf{C}(\mathbf{D})\right]_{ii} = (\left[\mathbf{I}\right]_i^* + \mathbf{d}_i)\mathbf{J}^{-1}
    (\left[\mathbf{I}\right]_i + \mathbf{d}_i^*) \quad \text{s.t. } \mathbf{d}_i \mathbf{W} \mathbf{d}_i^* 
    \leq \gamma_i
    \]
    \item The total variance is:
    \begin{align*}
    \min_{\mathbf{D}} \left\{ \sum_{i=1}^{m} (\left[\mathbf{I}\right]_i^* + \mathbf{d}_i)\mathbf{J}^{-1}(\left[\mathbf{I}\right]_i +
    \mathbf{d}_i^*) \right\}
    &= \min_{\mathbf{D}} \left\{ \text{Tr}\left((\mathbf{I}+\mathbf{D})\mathbf{J}^{-1}(\mathbf{I} +
    \mathbf{D})^*\right) \right\} \\
    &= \min_{\mathbf{D}} C(\mathbf{D})
    \end{align*}

    Thus, we have the same optimization problem!
    
\end{itemize}
\end{frame}

\begin{frame}{Comparison with UCRLB (cont.)}
\begin{itemize}
    \item Scalar UCRLB minimizes $\mathbf{C}(\mathbf{D})$ s.t. 
    \[[\mathbf{D}\mathbf{W}\mathbf{D}^*]_{ii} \leq \gamma_i, \quad 1 \leq i \leq m.\]
    \item Vector UCRLB minimizes $\mathbf{C}(\mathbf{D})$ s.t. 
    \[\sum_{i=1}^m[\mathbf{D}\mathbf{W}\mathbf{D}^*]_{ii} \leq \sum_{i=1}^m\gamma_i = \gamma, 
    \quad 1 \leq i \leq m.\]
    \item Scalar constraints are tighter for the same optimization problem.
    \item Joint optimization over \( \mathbf{D} \) yields lower overall variance
    \item Cross-correlation allows bias in one component to reduce total variance by compensating for another.
\end{itemize}
\end{frame}

%%%%%%%%%%%%%%%%%%%%%%%%%%%%%%%%%%%%%%%%%%%%%%%%%%%%%%%%%%%%%%%%%%%%%%%%%%%%%%%%%%%%%%%%%%%%%%%%

\begin{frame}{UCRLB with Worst-Case Bias Constraint}
We now consider the problem of minimizing $C(\mathbf{D})$ subject to
\[
\mathbf{z}^*\mathbf{S}\mathbf{D}^*\mathbf{D}\mathbf{S}\mathbf{z} \leq \gamma, \quad \mathbf{z} \in \mathbb{C}^m, \quad \mathbf{z}^*\mathbf{z} = 1.
\]
Let $\lambda_{max}$ the largest eigenvalue of $\mathbf{S}$.
\begin{itemize}
    \item If $\gamma \geq \lambda_{max}^2$, then we choose 
    $\mathbf{D} = -\mathbf{I} \Rightarrow C(\mathbf{D}) = 0$
    \item If $\gamma < \lambda_{max}^2$
\end{itemize}
\end{frame}




\begin{frame}{References}
  \scriptsize
  \begin{itemize}
    \item Hero et al., IEEE TSP, 1996
    \item Tikhonov, Sov. Math. Dokl., 1963
    \item Kay, Fundamentals of Statistical Signal Processing, 1993
    \item Eldar, IEEE TSP, 2004
    \item Fessler et al., IEEE Trans., multiple years
  \end{itemize}
\end{frame}

\begin{frame}{Q \& A}
  \centering
  Thank you for your attention! \\
  Questions and Discussion
\end{frame}

\end{document}
